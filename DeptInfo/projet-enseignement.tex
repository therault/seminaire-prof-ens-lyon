\documentclass[compress,aspectratio=169]{beamer}
\usepackage[utf8]{inputenc}
\usepackage{xcolor,soul}
%
\usepackage{xspace}
\usepackage[most]{tcolorbox}
\newtcolorbox{myblock}[2][]{beamer,title=#2,fonttitle=\sffamily,
  left=1mm,right=1mm,top=1mm,bottom=1mm,arc=2mm,
  colback=black,colupper=white,colframe=yellow,
  coltitle=black,title style={top color=red!70,bottom color=yellow},
  #1}

\mode<presentation>
{
  \usetheme{Replica}
  \setbeamercovered{transparent}
}

%% \AtBeginSection[]
%% {
%%   \begin{frame}<beamer>
%%     \frametitle{Outline}
%%      {\tiny
%%        \tableofcontents[sectionstyle=show/shaded,subsectionstyle=show/hide/hide]
%%     }
%%   \end{frame}
%% }

\title[Rencontre avec le Departement d'Informatique]{Projet d'Enseignement}
\subtitle{
  Laboratoire de l'Informatique du Parall\'elisme\\
  \'Ecole Normale Sup\'erieure de Lyon}

\author[herault@icl.utk.edu]{Thomas Herault, University of Tennessee, Knoxville}

\date[13 Mai 2024]{}

\begin{document}
\begin{frame}
  \titlepage
\end{frame}

\begin{frame}
     \frametitle{Outline}
     \tiny
        \tableofcontents
\end{frame}

\section{Cours de Syst\`eme d'Exploitation}

\begin{frame}
  \frametitle{Syst\`eme d'Exploitation}
  
  \begin{beamerboxesrounded}{But du cours}
    \begin{itemize}
    \item Vue d'ensemble du fonctionnement des syst\`emes d'exploitation
    \item Composants physiques d'un ordinateur, interactions, interface
    \item Repr\'esentation abstraite de la machine
    \item Compr\'ehension des co\^uts (m\'emoire ou temps) li\'es aux diff\'erentes op\'erations 
    \end{itemize}
  \end{beamerboxesrounded}
  
  \bigskip
  
  \small R\'ef\'erences:\\
  $[1]$ Silberschatz, Abraham, and Peter Baer Galvin. "Operatings System Concepts." (2013).\\
  $[2]$ Tanenbaum, Andrew. Modern operating systems. Pearson Education, Inc., 2009.\\
  $[3]$ Arpaci-Dusseau, Remzi H., and Andrea C. Arpaci-Dusseau. Operating systems: Three easy pieces. Arpaci-Dusseau Books, LLC, 2018.
  
\end{frame}

\begin{frame}
  \frametitle{D\'ecoupage du cours (Syst\`emes d'Exploitation)}

  \begin{columns}
    \begin{column}{.45\linewidth}
      \begin{description}
      \item[Introduction] {\footnotesize Histoire des OS, r\^oles, conceptions}
      \item[Processus] {\footnotesize Concept de processus, partage de temps, ordonnancement, communication inter-processus}
      \item[Synchronisation] {\footnotesize Gestion des interd\'ependances, exclusion mutuelle, s\'emaphores, priorit\'es}
      \item[M\'emoire] {\footnotesize M\'emoire virtuelle, pagination, politiques d'\'eviction, support mat\'eriel}
      \end{description}
    \end{column}\begin{column}{.45\linewidth}
      \begin{description}
      \item[Stockage] {\footnotesize Gestion des entr\'ees-sorties, interfaces, structures et fonctions internes}
      \item[Syst\`eme de fichiers] {\footnotesize Hi\'erarchie, r\'epertoire, volumes; journalisation, aggr\'egation et fiabilisation des ressources}
      \item[S\'ecurit\'e] {\footnotesize Droits d'acc\`es (fichiers, CPU, m\'emoire), isolation, contr\^ole d'acc\`es, redondance}
      \end{description}
    \end{column}
  \end{columns}

  \smallskip
  
  Sujets additionels: R\'eseaux et syst\`emes distribu\'es (TCP/IP); virtualisation
\end{frame}
  
\section{Projet Int\'egr\'e}

\begin{frame}
  \frametitle{Projet Int\'egr\'e}

  \begin{columns}
    \begin{column}{.45\linewidth}\small
      Buts:
      \begin{itemize}
      \item Acqu\'erir de l'exp\'erience en programmation imp\'erative (C, Python, C++, ...)
      \item Apprendre le g\'enie logiciel par la pratique (d\'efinition d'un cahier des charges, d'une interface, des m\'ethodes de test; avoir une conception modulaire d'un projet cons\'equent)
      \item Apprendre \`a travailler de mani\`ere collaborative (ma\^\i{}trise des outils logiciels -- git, gitlab, etc... -- mais aussi apprendre \`a travailler en groupe)
      \end{itemize}
    \end{column}\begin{column}{.45\linewidth}\small
      Mon exp\'erience dans le domaine:
      \begin{itemize}
      \item suivi de projets (TER) en tant que MdC
      \item dans le cadre de ma recherche : participation \`a plusieurs projets logiciels open source d\'evelop\'es en communaut\'es (PaRSEC 82,973 LOC, 20+ contributeurs; dplasma 90,844 LOC, 10+ contributeurs; TTG, 37,500 LOC, 6+ contributeurs; ULFM in Open MPI, 236,570 LOC, 50+ contributeurs)
      \end{itemize}
    \end{column}
  \end{columns}
\end{frame}

\section{Cours de Programmation Parall\`ele pour le M2}

\begin{frame}
  \frametitle{Programmation Parall\'ele: enjeux, outils, et recherche}

  \begin{beamerboxesrounded}{But du cours}
    Introduction \`a la programmation parall\'ele pour le calcul \`a haute performance (HPC). \'Etat de l'art des environnements mat\'eriel et logiciel, et introduction des probl\'ematiques de recherche l\'iees au HPC.
  \end{beamerboxesrounded}

  \bigskip
  
  \small R\'ef\'erences:\\
  $[1]$ Hager, Georg, and Gerhard Wellein. Introduction to high performance computing for scientists and engineers. CRC Press, 2010.\\
  $[2]$ Sterling, Thomas, Maciej Brodowicz, and Matthew Anderson. High performance computing: modern systems and practices. Morgan Kaufmann, 2017.\\
  $[3]$ Dongarra, Jack, Thomas Herault, and Yves Robert. Fault tolerance techniques for high-performance computing. Springer International Publishing, 2015.
  
\end{frame}

\begin{frame}
  \frametitle{D\'ecoupage du cours}
  \begin{columns}
    \begin{column}{.45\linewidth}
      \begin{itemize}
      \item Historique : quel type d'application, \'evolution des architectures
      \item Mod\`eles de programmation: MPI+OpenMP+CUDA
      \item Mod\`eles de programmation: Syst\`emes de t\^aches
      \item Simulation et calcul num\'erique -- alg\`bre lin\'eaire dense et creuse
      \end{itemize}
    \end{column}\begin{column}{.45\linewidth}
      \begin{itemize}
      \item Ordonnancement (jobs, t\^aches, communications)
      \item Tol\'erance aux fautes (approches g\'enerales)
      \item Tol\'erance aux fautes (approches sp\'ecifiques aux applications)
      \item Extension de MPI pour la tol\'erance aux fautes
      \end{itemize}
    \end{column}
  \end{columns}
\end{frame}

\section{Ce que je peux amener au d\'epartement}

\begin{frame}
  \frametitle{Ce que je peux amener au d\'epartement}

  \begin{columns}
    \begin{column}{.45\linewidth}
      \small
      Cours que je peux assurer ou dans lesquels je peux intervenir:
      \begin{itemize}
      \item Architecture des ordinateurs et syst\`eme d'exploitation (L3/1\`ere ann\'ee)
      \item R\'eseaux (L3/1\`ere ann\'ee)
      \item Programmation (L3/1\`ere ann\'ee)
      \item Projet Int\'egr\'e 1/2 (M1+M2)
      \item Algorithmique et Programmation Parall\`ele et Distribu\'ee (M1+M2)
      \item Syst\`emes Distribu\'es (M1+M2)
      \end{itemize}
    \end{column}\begin{column}{.45\linewidth}
      \small
      Ma recherche : \\
      Mod\'elisation et \'evaluation exp\'erimentale d'algorithmes parall\`eles et distribu\'es pour le HPC

      \bigskip
      
      Je peux faire d\'ecouvrir aux \'etudiants de l'ENS l'approche exp\'erimentale, et la mise en pratique des th\'eories enseign\'ees, dans un contexte de recherche, et les inciter \`a poursuivre en th\`ese sur ces aspects.

      \bigskip

      Je suis pr\^et \`a m'investir dans les charges administratives li\'ees au d\'epartement.
      
    \end{column}
  \end{columns}
  
\end{frame}

\end{document}
