\section{Research Program and Integration in ROMA}

\begin{frame}
  \frametitle{Task-based approaches $\Rightarrow$ sea-change in fault tolerance}

  \begin{center}
    \begin{tabular}{C{.4\linewidth}|C{.4\linewidth}}
      General-purpose FT & Application-specific FT \\\hline
      Coordinated checkpointing and rollback-recovery & Algorithm-Based Fault Tolerance  \\
      Replication                                     & Iterative convergence
    \end{tabular}

    \medskip
    
    \textcolor{blue}{HPC FT approaches have been designed for bulk-synchronous programming}
    
    \bigskip
  
    Task-based approach: asynchronous execution\\
    and dynamically scheduled independent tasks

    \bigskip
  
    \textcolor{red}{We need to redesign fault-tolerance in the context of task systems}
  \end{center}

\end{frame}

\frame{
  \frametitle{Opportunity and Challenge: Malleability}

  \begin{beamerboxesrounded}{Opportunity}
    \begin{center}
      \textcolor{red}{Task systems introduce generic malleability}
    \end{center}

    \begin{itemize}
    \item Tasks are units of work with identified inputs and outputs
    \item Runtime systems can migrate tasks to react to external condition changes (esp. process crash)
    \end{itemize}
  \end{beamerboxesrounded}

  Challenges:
  \begin{itemize}
  \item Non-linear performance degradation due to attrition
    \begin{itemize}
    \item[\frownie] Work and data placement is critical to algorithm performance and usually left to the programmer
    \item[\smiley] Crashed processes should be replaced, so performance degradation should only be transient
    \end{itemize}
  \item Redesign checkpointing as a per-task basis
    \begin{itemize}
    \item noncoordinated checkpointing and message logging
    \item selective replication
    \end{itemize}
  \end{itemize}
}

\frame{
  \frametitle{Applying ABFT to Task-Based Runtime Systems}

  \begin{itemize}
  \item ABFT: bulk-synchronous 'by nature'
    \begin{itemize}
    \item Need to keep the cheksum valid at each step
    \item Efficient \emph{because} checksum is updated by only extending synchronous operation
    \end{itemize}
  \item Task-Based Runtime Systems: asynchronous 'by nature'
    \begin{itemize}
    \item Add strict control flow to keep checksum up to date? \frownie
    \item How to react to failures? Conditional DAG?
    \end{itemize}
  \item ABFT + Tasks:
    \begin{itemize}
    \item Extend DAG with checksum / checksum update tasks
    \item Introduce scheduling priorities to make sure checksums are updated
    \item Define alternative paths to produce data, associate cost with paths
      \begin{itemize}
      \item A block can be computed by inversing the checksum
      \item but preferable path is to update directly
      \end{itemize}
    \end{itemize}
  \end{itemize}

  \begin{center}
    \textcolor{red}{Likely collaborations in NUMPEX}
  \end{center}
}

\begin{frame}
  \frametitle{Silent Data Corruption}

  SDCs are not well evaluated\footnote{In general we need more studies on faults in leadership systems and high-end systems}

  Intuition:
  \begin{itemize}
  \item They happen more often as the amount of memory grows
  \item They happen more often as the computational intensity grows
  \end{itemize}

  If the intuition is right, they will happen more and more frequently

  Application-specific fault tolerance can help
  \begin{itemize}
  \item Not necessarily (or not only) to correct failures
  \item But also to detect failures
  \end{itemize}

  \bigskip

  Challenge: Validation of computation is possible but \textcolor{red}{not reliable}

  \textcolor{red}{Ongoing work: how to (efficiently) use unreliable validation of work}

\end{frame}

\begin{frame}
  \frametitle{Fault tolerance as a tool for energy saving}

  \begin{itemize}
  \item Energy consumption of HPC centers becomes a problem (both economical and ecological)
    \begin{itemize}
    \item Current policy is to not bill allocation if a node within the allocation fails
    \item Some platforms are switching to bill-per-Joule
    \item Does it make sense to get 'free' Joules thanks to a node failure?
    \item[] ~
    \item Cloud-HPC convergence: CPUs are billed as soon as allocated, even if the application cannot be deployed
    \end{itemize}
  \item Using stranded energy in Cloud or HPC
    \begin{itemize}
    \item Non-dispatchable renewable energy sources to power HPC
    \item $\Rightarrow$ Power cost varies with time (soft version)
    \item $\Rightarrow$ Power capping varies with time (hard version)
    \item $\Rightarrow$ Job scheduler may have to kill apps. Resilience can help.
    \end{itemize}
  \end{itemize}
  \begin{center}
    \textcolor{red}{Fits well the Energy Minimization topic in ROMA}
  \end{center}
\end{frame}


