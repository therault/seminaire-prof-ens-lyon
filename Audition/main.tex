\documentclass[compress,aspectratio=169]{beamer}
\usepackage[utf8]{inputenc}
\usepackage{xcolor,soul}
%
\usepackage{amsmath}
\usepackage{amsfonts}
\usepackage{amssymb}
\usepackage[american]{babel}
\usepackage{xspace}
\usepackage{graphicx}
\usepackage{ifthen}
\usepackage{figlatex}
\usepackage{colortbl}
\usepackage[labelformat=empty]{caption}
\renewcommand{\captionfont}{\scriptsize}
\usepackage{listings}
\usepackage{fancyvrb}
\usepackage[absolute,overlay]{textpos}
\usepackage{environ}
\usepackage{dot2texi}
\usepackage{tipa}
\usepackage{pifont,ulsy}
\usepackage{mhchem}
\usepackage{relsize}
\usepackage{array}

\newcolumntype{C}[1]{>{\centering\let\newline\\\arraybackslash\hspace{0pt}}m{#1}}

\usepackage{tabularx}

\newcommand{\myitem}[1]{%
\begin{tabularx}{\linewidth}{@{}l@{}>{\raggedright}X@{}}
\usebeamercolor[fg]{structure}$\bullet$~~ & #1
\end{tabularx}%
}    
\newenvironment{myitemize}{\par}{\par}   

\AtBeginDocument{\renewcommand{\proofname}{{\bf Proof}}}

\makeatletter
\DeclareOptionBeamer{compress}{\beamer@compresstrue}
\ProcessOptionsBeamer

\makeatletter
\let\@@magyar@captionfix\relax
\makeatother

\graphicspath{{figures/}{fig}}

\definecolor{cinnabar}{rgb}{1,0.5,0.5}
\makeatletter
\newcommand\SoulColor{%
  \let\set@color\beamerorig@set@color
  \let\reset@color\beamerorig@reset@color}
\makeatother
\usepackage[most]{tcolorbox}

\newtcolorbox{myblock}[2][]{beamer,title=#2,fonttitle=\sffamily,
  left=1mm,right=1mm,top=1mm,bottom=1mm,arc=2mm,
  colback=black,colupper=white,colframe=yellow,
  coltitle=black,title style={top color=red!70,bottom color=yellow},
  #1}


\RecustomVerbatimCommand{\VerbatimInput}{VerbatimInput}{fontsize=\scriptsize}

\mode<presentation>
{
  \usetheme{Replica}
  \setbeamercovered{transparent}
}

\makeatletter
\let\beamer@writeslidentry@miniframeson=\beamer@writeslidentry
\def\beamer@writeslidentry@miniframesoff{%
  \expandafter\beamer@ifempty\expandafter{\beamer@framestartpage}{}% does not happen normally
  {
    \clearpage\beamer@notesactions%
  }
}
\newcount\beamer@writeslidentry@miniframesoneoversome@counter
\newcount\beamer@writeslidentry@miniframesoneoversome@limit
\beamer@writeslidentry@miniframesoneoversome@counter=1
\newcount\beamer@writeslidentry@miniframesoneoversome@limit
\beamer@writeslidentry@miniframesoneoversome@limit=5
\def\beamer@writeslidentry@miniframesoneoversome{%
  \ifnum\beamer@writeslidentry@miniframesoneoversome@counter=\beamer@writeslidentry@miniframesoneoversome@limit\relax%
    \beamer@writeslidentry@miniframesoneoversome@counter=1\relax%
    \beamer@writeslidentry@miniframeson%
  \else%
    \advance\beamer@writeslidentry@miniframesoneoversome@counter by 1%
    \beamer@writeslidentry@miniframesoff%
  \fi%
}
\newcommand*{\miniframeson}{\let\beamer@writeslidentry=\beamer@writeslidentry@miniframeson}
\newcommand*{\miniframesoff}{\let\beamer@writeslidentry=\beamer@writeslidentry@miniframesoff}
\newcommand*{\miniframesalternate}[1]{%
   \beamer@writeslidentry@miniframesoneoversome@counter=1\relax%
   \beamer@writeslidentry@miniframesoneoversome@limit=#1\relax%
   \let\beamer@writeslidentry=\beamer@writeslidentry@miniframesoneoversome}

% Define the subsectionsonly option for beamer toc
\def\beamer@tocaction@only#1{\only<.(1)>{\usebeamertemplate**{#1}}}
\define@key{beamertoc}{subsectionsonly}[]{\beamer@toc@subsectionstyle{show/only}\beamer@toc@subsubsectionstyle{show/shaded/hide}}
\makeatother

\usepackage{enumerate}
\usepackage{url,xspace}
\usepackage{relsize}
\usepackage[ruled,vlined]{algorithm2e}
\usepackage{subfigure}
\usepackage{graphicx,psfrag}
\usepackage{multirow}
\usepackage{rotating}
\usepackage{pstool}
\usepackage{blkarray}
\usepackage{wasysym}
\usepackage{tikzsymbols}

\def\smiley{\green{\larger[2]\wasyfamily\char44}\xspace}
\def\frownie{\blue{\larger[2]\wasyfamily\char47}\xspace}
\def\quesley{\yyellow{\Neutrey}\xspace}

\usepackage{tipa}
\usepackage{pifont}

\AtBeginDocument{\renewcommand{\proofname}{{\bf Proof}}}
\newcommand{\itemb}{\item[$\bullet$]}
 \newcommand{\un}{1 \hspace{-0.132cm}1}
\newcommand{\ignore}[1]{}

\newcommand<>{\blue}[1]{{\color#2{blue!100!black!100}#1}}
\newcommand<>{\red}[1]{{\color#2{red!80!black}#1}}
\newcommand<>{\green}[1]{{\color#2{green!70!black}#1}}
\newcommand<>{\purple}[1]{{\color#2{blue!50!red}#1}}
\newcommand<>{\yyellow}[1]{{\color#2{yellow!90!black}#1}}

\usepackage{tcolorbox}
\tcbuselibrary{theorems}
\tcbuselibrary{skins}
\tcbuselibrary{fitting}
\newtcolorbox{mythmbox}[2][]{colback=white,fonttitle=\bfseries,enhanced,attach boxed title to top left={xshift=15mm,yshift=-2mm},title=#2,#1}
\newtcbox{mybox}[2][]{fonttitle=\bfseries,enhanced,attach boxed title to top left={xshift=15mm,yshift=-2mm},title=#2,#1}

\usepackage{tikz}
\usetikzlibrary{patterns}
\usetikzlibrary{arrows,shapes}
\usetikzlibrary{shapes.geometric}
\usetikzlibrary{decorations.pathreplacing, calc}

\tcbuselibrary{theorems}
\tcbuselibrary{skins}
\tcbuselibrary{fitting}

\usetikzlibrary{patterns}
\usetikzlibrary{arrows,shapes}
\usetikzlibrary{shapes.geometric}
\usetikzlibrary{decorations.pathreplacing, calc}
\usetikzlibrary{calc}
\usetikzlibrary{arrows,shapes}
\usetikzlibrary{patterns,snakes}
\usetikzlibrary{babel}
\usetikzlibrary{scopes,intersections}
\usetikzlibrary{decorations.pathreplacing}

\usetikzlibrary{patterns,arrows,arrows.meta,shapes,decorations.markings,tikzmark,matrix,fit,positioning,decorations.pathreplacing,calc}
\tikzstyle{vecArrow} = [thick, decoration={markings,mark=at position
   1 with {\arrow[semithick]{open triangle 60}}},
   double distance=1.4pt, shorten >= 5.5pt,
   preaction = {decorate},
   postaction = {draw,line width=1.4pt, white,shorten >= 4.5pt}]
\tikzstyle{innerWhite} = [semithick, white,line width=1.4pt, shorten >= 4.5pt]


\pgfdeclarelayer{background}
\pgfdeclarelayer{foreground}
\pgfsetlayers{background,main,foreground}

\usepackage{pgfplots}

\newcommand*{\boxcolor}{red}
\makeatletter
\renewcommand{\boxed}[1]{\textcolor{\boxcolor}{%
\tikz[baseline={([yshift=-1ex]current bounding box.center)}] \node [rectangle, minimum width=1ex,rounded corners,draw] {\normalcolor\m@th$\displaystyle#1$};}}
\makeatother

\makeatletter
\newsavebox{\measure@tikzpicture}
\NewEnviron{scaletikzpicturetowidth}[1]{%
  \def\tikz@width{#1}%
  \def\tikzscale{1}\begin{lrbox}{\measure@tikzpicture}%
  \BODY
  \end{lrbox}%
  \pgfmathparse{#1/\wd\measure@tikzpicture}%
  \edef\tikzscale{\pgfmathresult}%
  \BODY
}
\makeatother

\usepackage{tikz-timing}
\usetikztiminglibrary{beamer}
\usepackage{sidecap}
\usepackage{tikz}
\usetikzlibrary{calc}
\usetikzlibrary{arrows,shapes}
\usetikzlibrary{patterns,snakes}
\usetikzlibrary{babel}
\usetikzlibrary{scopes,intersections}
\usetikzlibrary{decorations.pathreplacing}

\pgfdeclarelayer{background}
\pgfdeclarelayer{foreground}
\pgfsetlayers{background,main,foreground}



%Macros for prediction drawings.
\newcommand{\fault}[1]{
\draw[->, color=red] ($(#1)+(0.2,1)$) -- ($(#1)+(0.1,0.7)$) -- ($(#1)+(0.2,0.6)$) -- (#1);
} %donner les coordonnées de l'endroit où la faute arrive.

\newcommand{\faultfailstop}[1]{
\draw[<-, color=red] (#1) -- ($(#1)+(0.2,1.2)$) -- ($(#1)+(0.1,1.4)$) --  ($(#1)+(0.2,2.2)$) node[above, right] {\scriptsize{fault}};
} %donner les coordonnées de l'endroit où la faute arrive.

\newcommand{\faultsilent}[1]{
\draw[<-, color=red] (#1) -- ($(#1)+(0.2,1.2)$) -- ($(#1)+(0.1,1.4)$) --  ($(#1)+(0.2,2.2)$) node[above] {\scriptsize{silent error}};
} %donner les coordonnées de l'endroit où la faute arrive.

\newcommand{\falsealarm}[1]{
\draw[<-, color=blue] (#1) -- ($(#1)+(0.2,0.8)$) -- ($(#1)+(0.1,1)$) --  ($(#1)+(0.2,1.6)$)  node[above, right] {\scriptsize{false alarm}};
}

\newcommand{\detecfault}[1]{

}


\newcommand{\faultbis}[1]{
\draw[<-, color=red] (#1) -- ($(#1)+(0.2,1.2)$) -- ($(#1)+(0.1,1.4)$) --  ($(#1)+(0.2,2)$) node[above, left] {\scriptsize{fault}};
} %donner les coordonnées de l'endroit où la faute arrive.
\newcommand{\smallpred}[1]{
\draw[<-, color=blue] (#1) -- ($(#1)+(0.2,1.2)$) -- ($(#1)+(0.1,1.4)$) --  ($(#1)+(0.2,2)$)  node[above, right] {\scriptsize{pred.}};
} 
\newcommand{\faultpred}[1]{
\draw[<-, color=green] (#1) -- ($(#1)+(0.2,1.2)$) -- ($(#1)+(0.1,1.4)$) --  ($(#1)+(0.2,2)$)  node[above] {\scriptsize{F+P}};
} 


\newcommand{\detecfaultyes}[1]{
\draw[<-, color=blue] (#1) -- ($(#1)+(0.7,1)$) -- ($(#1)+(0.6,1.2)$) --  ($(#1)+(1.3,2)$)  node[above, right] {\scriptsize{detect!}};
}
\newcommand{\detecfaultno}[1]{
\draw[<-, color=blue] (#1) -- ($(#1)+(0.2,0.8)$) -- ($(#1)+(0.1,1)$) --  ($(#1)+(0.2,1.6)$)  node[above, right] {\scriptsize{detect?}};
}

\newcommand{\corruptckpsure}[1]{
\draw[thick, color=red] ($(#1)+(-0.6,-1)$) -- ($(#1)+(0.6,1)$) ($(#1)+(0.6,-1)$) -- ($(#1)+(-0.6,1)$)  node at ($(#1)+(0,1.5)$) {\scriptsize{corrupted!}};
}

\newcommand{\corruptckpnotsure}[1]{
\draw[thick, color=orange] ($(#1)+(-0.6,-1)$) -- ($(#1)+(0.6,1)$) ($(#1)+(0.6,-1)$) -- ($(#1)+(-0.6,1)$)  node at ($(#1)+(0,1.5)$) {\scriptsize{corrupted?}};
}

\newcommand{\predfault}[1]{
\draw[<-, color=blue] ($(#1) + (0,1)$) -- ($(#1)+(0.2,1.6)$)  -- ($(#1)+(0.1,1.7)$) -- ($(#1)+(0.2,2)$)  node[above, right] {\scriptsize{Predicted fault}};
} 

\newcommand{\predfaultint}[2]{
\draw[thick, color=blue,<->] ($(#1)+(0,1.10)$) -- ($(#1)+(#2,1.10)$) node[above=-0.5pt, midway] {\scriptsize{\I}};
\draw[dashed, color=blue] ($(#1)$) --  ($(#1)+(0,1.10)$);
\draw[dashed, color=blue] ($(#1)+(#2,0)$) --  ($(#1)+(#2,1.10)$);
} %Le deuxieme argument est la longueur de l'intervalle I, le premier est l'endroit où il commence.
\newcommand{\faultint}[2]{
\draw[thick, color=red] ($(#1)+(0,0.10)$) -- ($(#1)+(#2,0.10)$);
}

\newcommand{\simplytextbelow}[3]{
\draw[thick,color=white] ($(#1)+(0,-0.30)$) -- ($(#1)+(#2,-0.30)$) node[below=0.0pt, midway] {\scriptsize{#3}};
}

\newcommand{\simplytextabove}[3]{
\draw[thick,color=white] ($(#1)+(0,0.30)$) -- ($(#1)+(#2,0.30)$) node[below=0.0pt, midway] {\scriptsize{#3}};
}

\newcommand{\legend}[3]{
  \draw[thin, <->] ($(#1)+(0,-0.20)$) -- ($(#1)+(#2,-0.20)$) node[below=-0.5pt, midway] {\scriptsize{#3}};
}

\newcommand{\legende}[3]{
\draw[thick, <->] ($(#1)+(0,-0.20)$) -- ($(#1)+(#2,-0.20)$) node[below=-0.5pt, midway] {\scriptsize{#3}};
}

\newcommand{\semilegende}[3]{
\draw[thick,dashed,<-] ($(#1)+(0,-0.20)$) -- ($(#1)+(1,-0.20)$) node[below=-0.5pt, midway] {};
\draw[thick,->] ($(#1)+(1,-0.20)$) -- ($(#1)+(#2,-0.20)$) node[below=-0.5pt, midway] {\scriptsize{#3}};
}
\newcommand{\legendeW}[3]{
\draw[thick, <->] ($(#1)+(0,-0.20)$) -- ($(#1)+(#2,-0.20)$) node[below=+0.5pt, midway] {\scriptsize{#3}};
}

\newcommand{\legendelongue}[4]{
\draw[thick,<->] ($(#1)+(0,-0.20)$) -- ($(#1)+(#2,-0.20)$) node[below=-0.5pt, midway] {\scriptsize{#3}};
\draw[draw=none] ($(#1)+(0,-0.80)$) -- ($(#1)+(#2,-0.80)$) node[below=-0pt, midway] {\scriptsize{#4}};
}


\newcommand{\arrowtime}[2]{
\draw[thick, color=black,->] (0,#1) -- (#2,#1) node[below=-0.5pt, ] {\scriptsize{Time}};
}

\newcommand{\rd}{0.4cm}
\newcommand{\ttrd}{4} %time table row distance

\newcommand{\bracetikz}[2]
{\draw [thick,color=red, decorate,decoration={brace,amplitude=10pt,mirror},xshift=0.4pt,yshift=-0.4pt] ($(#1)+(0,-0.2)$) -- ($(#2)+(0,-0.2)$) node[black,midway,yshift=-0.6cm] {};}

\newcommand{\patternCV}[3]{
\draw[very thick, color=red] ($(#1,#3)+(0,-0.5)$) -- ($(#1,#3)+(0,1.5)$);
\draw[very thick, color=red] ($(#2,#3)+(0,-0.5)$) -- ($(#2,#3)+(0,1.5)$);
}

\newcommand{\patternCVlong}[3]{
\draw[very thick, color=red] ($(#1,#3)+(0,-1.5)$) -- ($(#1,#3)+(0,1.8)$);
\draw[very thick, color=red] ($(#2,#3)+(0,-1.5)$) -- ($(#2,#3)+(0,1.8)$);
}

\newcommand{\patternCVlooong}[3]{
\draw[very thick, color=red] ($(#1,#3)+(0,-2)$) -- ($(#1,#3)+(0,1.8)$);
\draw[very thick, color=red] ($(#2,#3)+(0,-2)$) -- ($(#2,#3)+(0,1.8)$);
}

\newcommand{\here}[2]{
\draw[very thick,<-] ($(#1,#2)$) -- ($(#1,#2)+(0,-1.5)$);
}

\newcommand{\failstop}[1]{
\draw[<-, color=red] (#1) -- ($(#1)+(0.2,1.2)$) -- ($(#1)+(0.1,1.4)$) --  ($(#1)+(0.2,2.2)$) node[above, left] {\scriptsize{Fail-stop Error}};
}

\newcommand{\proc}[3][white]{
%\draw[thick, color=black,->] (0,#1) -- (#2,#1) node[below=-0.5pt, ] {\scriptsize{Time}};
\draw (-0.5,#3) node [fill=#1,draw, circle, inner sep = 1pt] {\scriptsize{\ema{#2}}};
}


\newcommand\blfootnote[1]{%
  \begingroup
  \renewcommand\thefootnote{}\footnote{#1}%
  \addtocounter{footnote}{-1}%
  \endgroup
}

%clock
\usepackage[font=Helv,timeinterval=60]{tdclock}

\title[Candidature Professeur ENS Lyon]{Thomas Herault}

\subtitle{Candidature Professeur ENS Lyon}

\author[herault@icl.utk.edu]{Thomas Herault, University of Tennessee, Knoxville}

\date[May 15, 2024]{}

\tikzset{
  title overlay node/.style={
    draw=white,fill=white,rounded corners,anchor=north west,
  },
}
\def\tikzoverlay{%
   \tikz[baseline,overlay]\node[title overlay node]
}%

\beamertemplatenavigationsymbolsempty

\begin{document}

%\miniframesoff

\begin{frame}<2>[t]
  \maketitle

    \pgfplotsset{
        /pgf/number format/year/.style={fixed, 1000 sep={}}
    }
    \only<1>{
      \def\tlopacity{0}
      \def\bopacity{0}
    }
    \only<2>{
      \def\tlopacity{1}
      \def\bopacity{0.2}
    }

    \resizebox{\textwidth}{!}{
    \begin{tikzpicture}
        \draw[->,line width=.2mm,ultra thick,opacity=\tlopacity] (0,0) -- (25,0);
        \foreach \x in {0,...,24}{\draw[thick,opacity=\tlopacity] (\x,0.2)--(\x,-0.2);}
        \foreach \x in {0,4,...,24}{\pgfmathsetmacro\result{\x + 2000}\node[opacity=\tlopacity] at (\x,-0.4) {\pgfmathprintnumber[year]{\result}};}
    
        \fill[rounded corners=10pt,fill=black,fill opacity=\bopacity] (0,0.3) rectangle (3,-0.3);
        \node[align=center,text width=3cm,opacity=\tlopacity] at (1.5,1) {Moniteur Normalien (Paris XI)};

        \fill[rounded corners=10pt,fill=black,fill opacity=\bopacity] (3,0.3) rectangle (4,-0.3);
        \node[align=center,text width=2cm,opacity=\tlopacity] at (3.5,1) {ATER (Paris XI)};

        \fill[rounded corners=10pt,fill=black,fill opacity=\bopacity] (4,0.3) rectangle (24,-0.3);
        \node[align=center,text width=8cm,opacity=\tlopacity] at (14,1) {Ma\^\i{}tre de Conf\'erences (Paris XI)};

        \fill[rounded corners=10pt,fill=black,fill opacity=\bopacity] (8,-0.7) rectangle (10,-1.3);
        \node[align=center,text width=3cm,opacity=\tlopacity] at (9,-1) {D\'el. INRIA};

        \fill[rounded corners=10pt,fill=black,fill opacity=\bopacity] (10,-0.7) rectangle (20,-1.3);
        \node[align=center,text width=6cm,opacity=\tlopacity] at (15,-1) {D\'etachement UTK};

        \fill[rounded corners=10pt,fill=black,fill opacity=\bopacity] (20,-0.7) rectangle (24,-1.3);
        \node[align=center,text width=3cm,opacity=\tlopacity] at (22,-1) {Disponibilit\'e};

        \fill[rounded corners=10pt,fill=black,fill opacity=\bopacity] (8,-1.5) rectangle (10,-2.1);
        \node[align=center,text width=3cm,opacity=\tlopacity] at (9,-1.8) {Vis. Sch. (UTK)};

        \fill[rounded corners=10pt,fill=black,fill opacity=\bopacity] (10,-1.5) rectangle (17,-2.1);
        \node[align=center,text width=6cm,opacity=\tlopacity] at (13.5,-1.8) {Research Scientist II (UTK)};

        \fill[rounded corners=10pt,fill=black,fill opacity=\bopacity] (17,-1.5) rectangle (20,-2.1);
        \node[align=center,text width=3cm,opacity=\tlopacity] at (18.5,-1.8) {Res. Dir. (UTK)};
    
        \fill[rounded corners=10pt,fill=black,fill opacity=\bopacity] (20,-1.5) rectangle (24,-2.1);
        \node[align=center,text width=4cm,opacity=\tlopacity] at (22,-1.8) {Res. Ass. Pr. (UTK)};
    \end{tikzpicture}}
  \end{frame}
        
%%
%% TODO: faire apparaitre les problematiques de recherche
%%

\frame{
    \frametitle{Background}

    \pgfplotsset{
        /pgf/number format/year/.style={fixed, 1000 sep={}}
    }

    \resizebox{\textwidth}{!}{
    \begin{tikzpicture}
        \draw[->,line width=.2mm,ultra thick] (0,0) -- (25,0);
        \foreach \x in {0,...,24}{\draw[thick] (\x,0.2)--(\x,-0.2);}
        \foreach \x in {0,4,...,24}{\pgfmathsetmacro\result{\x + 2000}\node at (\x,-0.4) {\pgfmathprintnumber[year]{\result}};}

        \fill[rounded corners=10pt,fill=black,fill opacity=0.2] (0,0.3) rectangle (3,-0.3);
        \node[align=center,text width=3cm] at (1.5,1) {Moniteur Normalien (Paris XI)};

        \fill[rounded corners=10pt,fill=black,fill opacity=0.2] (3,0.3) rectangle (4,-0.3);
        \node[align=center,text width=2cm] at (3.5,1) {ATER (Paris XI)};

        \fill[rounded corners=10pt,fill=black,fill opacity=0.2] (4,0.3) rectangle (24,-0.3);
        \node[align=center,text width=8cm] at (14,1) {Ma\^\i{}tre de Conf\'erences (Paris XI)};

        \fill[rounded corners=10pt,fill=black,fill opacity=0.2] (8,-0.7) rectangle (10,-1.3);
        \node[align=center,text width=3cm] at (9,-1) {D\'et. INRIA};

        \fill[rounded corners=10pt,fill=black,fill opacity=0.2] (10,-0.7) rectangle (20,-1.3);
        \node[align=center,text width=6cm] at (15,-1) {D\'etachement UTK};

        \fill[rounded corners=10pt,fill=black,fill opacity=0.2] (20,-0.7) rectangle (24,-1.3);
        \node[align=center,text width=3cm] at (22,-1) {Disponibilit\'e};

        \fill[rounded corners=10pt,fill=black,fill opacity=0.2] (8,-1.5) rectangle (10,-2.1);
        \node[align=center,text width=3cm] at (9,-1.8) {Vis. Sch. (UTK)};

        \fill[rounded corners=10pt,fill=black,fill opacity=0.2] (10,-1.5) rectangle (17,-2.1);
        \node[align=center,text width=6cm] at (13.5,-1.8) {Research Scientist II (UTK)};

        \fill[rounded corners=10pt,fill=black,fill opacity=0.2] (17,-1.5) rectangle (20,-2.1);
        \node[align=center,text width=3cm] at (18.5,-1.8) {Res. Dir. (UTK)};
    
        \fill[rounded corners=10pt,fill=black,fill opacity=0.2] (20,-1.5) rectangle (24,-2.1);
        \node[align=center,text width=4cm] at (22,-1.8) {Res. Ass. Pr. (UTK)};

        \fill[rounded corners=10pt,fill=red,fill opacity=0.2] (0,-2.5) rectangle (6,-3.1);
        \node[align=center,text width=6cm] at (3,-2.8) {Self-Stabilization, Dist. Alg.};

        \fill[rounded corners=10pt,fill=black,fill opacity=0.2] (7,-2.5) rectangle (11,-3.1);
        \node[align=center,text width=6cm] at (9,-2.8) {``D\'efi S\'ecurit\'e''};

        \fill[rounded corners=10pt,fill=purple,fill opacity=0.2] (13,-2.5) rectangle (24,-3.1);
        \node[align=center,text width=12cm] at (18.5,-2.8) {Performance Modeling for Fault Tolerance};
       
        \fill[rounded corners=10pt,fill=green,fill opacity=0.2] (3,-3.3) rectangle (10,-3.9);
        \node[align=center,text width=6cm] at (6.5,-3.6) {Rollback recovery in MPICH};

        \fill[rounded corners=10pt,fill=brown,fill opacity=0.2] (10,-3.3) rectangle (24,-3.9);
        \node[align=center,text width=18cm] at (17,-3.6) {Task-Based Runtime Systems -- Prog. Inter.; Applications; Mem. Const. Alg.};

        \fill[rounded corners=10pt,fill=blue,fill opacity=0.2] (8,-4.1) rectangle (24,-4.7);
        \node[align=center,text width=12cm] at (16,-4.4) {Fault Tolerant MPI, Application-Specific FT for HPC};

        \fill[rounded corners=10pt,fill=black,fill opacity=0.2] (2,-4.1) rectangle (7,-4.7);
        \node[align=center,text width=6cm] at (4.5,-4.4) {App. Prob. Model Checking};
    \end{tikzpicture}}
}   

\frame{
  \frametitle{My contributions to HPC (overview)}

  Two complementary approaches to Fault Tolerance
  \begin{itemize}
  \item General Purpose Fault Tolerance
    \begin{itemize}
    \item Checkpointing and rollback-recovery
      \begin{itemize}
      \item Experimental approach: implementing new protocols and evaluating them in 'real' applications / deployments
      \item Performance models approach: design mathematical models to project performance at scale or under new assumptions
      \end{itemize}
    \end{itemize}
  \item Application-Specific Fault Tolerance
    \begin{itemize}
    \item Design / adapt communication middlewares to write fault-tolerant applications
    \item Design new fault-tolerance schemes for parallel applications
    \end{itemize}
  \end{itemize}

  Task-Based Runtime Systems and Fault Tolerance
  \begin{itemize}
  \item New runtime systems and programming interfaces for modern hardware architectures
  \item Design fault-tolerant applications over TBRS
  \end{itemize}
}

\frame{
  \frametitle{Overview: General Purpose Fault Tolerance}

  \begin{columns}
    \begin{column}{.5\textwidth}
      Experimental approach: MPICH-V
      \begin{itemize}
      \item Coordinated Rollback-Recovery
        \begin{itemize}
        \item Blocking/Non Blocking
        \end{itemize}
      \item Uncoordinated Rollback-Recovery
        \begin{itemize}
        \item Message Logging
        \item Optimistic / Pessimistic / Causal
        \end{itemize}
      \item Hierarchical Approach
      \end{itemize}
    \end{column}
    \begin{column}{.5\textwidth}
      Performance Models for Rollback-Recovery
      \begin{itemize}
      \item \textcolor{blue}{Young-Daly}
      \item \textcolor{blue}{Hierarchical Checkpointing}
      \item In-Memory Checkpointing
      \item Managing I/O Contention
      \end{itemize}
    \end{column}
  \end{columns}
}

\frame{
  \frametitle{Overview: Application-Specific Fault-Tolerance}
  
  \begin{columns}
    \begin{column}{.5\textwidth}
      User-Level Failure Mitigation
      \begin{itemize}
      \item Specification / Standardization
      \item \textcolor{blue}{Failure Detection}
      \item \textcolor{blue}{Consensus}
      \item Silent Errors
      \end{itemize}
    \end{column}
    \begin{column}{.5\textwidth}
      Roll-forward fault-tolerance
      \begin{itemize}
      \item ABFT for dense matrix factorization
      \item Composite Approach: ABFT \& Checkpointing
      \item Moldable Applications
      \end{itemize}
    \end{column}
  \end{columns}
}

\frame{
  \frametitle{Overview: Task-Based Runtime Systems}
  
  \begin{columns}
    \begin{column}{.5\textwidth}
      Parallel Runtime Scheduler and Execution Controller (PaRSEC)
      \begin{itemize}
      \item Micro task (task scheduling overhead $\sim 200ns$)
      \item Distributed (runs with thousands of nodes)
      \item Hybrid (Intel, AMD, NVIDIA GPUs)
      \item Implicit data movement (from device to device, in the background)
      \item Multiple programming interfaces
        \begin{itemize}
        \item Parameterized Task Graphs
        \item Dynamic Task Discovery
        \item Template Task Graphs
        \end{itemize}
      \end{itemize}
    \end{column}
    \begin{column}{.5\textwidth}
      \begin{itemize}
      \item Applications:
        \begin{itemize}
        \item Dense Linear Algebra (DPLASMA)
        \item Sparse Linear Algebra (PaStiX, TiledArray)
        \item Quantum Chemistry (NWChem, MPQC)
        \item Multi-representation (HiCMA)
        \end{itemize}
      \item Node-to-node migratable tasks
      \item Resilient extensions
        \begin{itemize}
        \item Resilient to Silent Data Corruptions via Algorithm-Based Fault-Tolerance validation
        \item Task-based checkpoint and partial rollback-recovery
        \end{itemize}
      \end{itemize}
    \end{column}
  \end{columns}
}

\section[Fault Tolerance]{Theme 1: Fault Tolerance}

\begin{frame}
  \frametitle{Research Topic 1: Fault Tolerance}

  Two complementary approaches to \textcolor{blue}{Fault Tolerance}

  \bigskip

  \begin{tabular}{p{.45\linewidth}|p{.45\linewidth}}
    General Purpose Fault Tolerance     & Application-Specific Fault Tolerance \\
    &\\\hline
    &\\
    \begin{minipage}{\linewidth}
      \begin{itemize}
      \item Checkpointing and rollback-recovery
      \item How to checkpoint (protocols)
      \item When to checkpoint
      \end{itemize}
    \end{minipage} & \begin{minipage}{\linewidth}
      \begin{itemize}
      \item Middleware to enable writing FT applications
      \item Resilient algorithms
      \item How to build a resilient application
      \end{itemize}
    \end{minipage}
  \end{tabular}
\end{frame}


\begin{frame}
  \frametitle{Major Contribution 1: Enabling fault-tolerant MPI applications}

  \begin{columns}
    \begin{column}{.4\textwidth}
      \begin{minipage}{\linewidth}
        \begin{center}
          \includegraphics[width=.9\linewidth]{MPIlogo.jpg}
          
          Message Passing Interface
        \end{center}
        \begin{itemize}
        \item De-facto standard of parallel programming in HPC
        \item $>99\%$ of HPC applications
        \item Available on all HPC systems
          \begin{itemize}
          \item vendor implementations based on either Open MPI or MPICH
          \end{itemize}
        \end{itemize}
      \end{minipage}
    \end{column}\begin{column}{.7\linewidth}
      Support for process failures in MPI:
      \begin{itemize}
      \item Until 2020 no support to tolerate failures\\
        Process crash $\Rightarrow$ communication failure \\
        communication failure $\Rightarrow$ MPI error \\
        MPI error $\Rightarrow$ interruption of the whole application / undefined behavior
      \item User-Level Failure Mitigation:\\
        \begin{itemize}
        \item A proposal to change the MPI standard to enable writing fault-tolerant applications
        \item An experimental implementation based on Open MPI
        \item Leading ideas:
          \begin{itemize}
          \item Minimal change of the MPI specification
          \item Addition of resilient features: failure detection, group membership, reliable broadcast, consensus
          \end{itemize}
        \end{itemize}
      \end{itemize}      
    \end{column}
  \end{columns}
  
\end{frame}

\begin{frame}
  \frametitle{My contributions related to ULFM}
  
  My major contributions related to ULFM:
  \begin{itemize}
  \item Co-designer of the ULFM specification \\
    $\Longrightarrow$\textcolor{red}{MPI Standard}
  \item Routines for the implementation of ULFM
    \begin{itemize}
    \item Failure Detector ($\Rightarrow$\textcolor{red}{best paper finalist SC'16})
    \item Consensus routine 
    \end{itemize}
  \item New application-specific fault-tolerant algorithms
    \begin{itemize}
    \item ABFT algorithms for matrix factorizations
    \item Managing process attrition in moldable applications
    \end{itemize}
  \item New composition technique of fault-tolerant approaches
  \end{itemize}

  ULFM software available by default on all HPC systems -- \textcolor{red}{integration in MPICH and Open MPI}
  
  Significant impact in the research community: $>10$ external projects depend on ULFM; highly cited research
\end{frame}

\section[Task-Based Runtime Systems]{Theme 2: Task-Based Runtime Systems}

\begin{frame}
  \frametitle{Research Topic 2: Task-Based Runtime Systems for HPC}

  \textcolor{brown}{Task Based Runtime Systems}
  \begin{itemize}
  \item Main question: how to reach performance portability in modern HPC architectures
    \begin{itemize}
    \item Networks with deep hierarchies / Manycore / NUMA / Accelerators
    \item Separation of concerns: going beyond MPI+X
    \item Management of asynchrony
    \item Overlap of computations and data movement
    \end{itemize}
  \item Related topics:
    \begin{itemize}
    \item Expression of parallelism / Programming Interfaces
    \item Out-of-(accelerator) core memory programming / Memory-Constrained algorithms
    \item Multiresolution computation with dynamic decision
    \item Sparse and computation-dependent task systems
    \end{itemize}
  \end{itemize}

\end{frame}


\begin{frame}
  \frametitle{Major Contribution 2: Task-Based Runtime Systems}

  \begin{columns}
    \begin{column}{.4\textwidth}
      \begin{minipage}{\linewidth}
        \begin{center}
          \includegraphics[width=.9\linewidth]{../parsec.pdf}
          
          \scriptsize PaRSEC: Parallel Runtime Scheduler and Execution Control
        \end{center}
        \begin{itemize}\scriptsize
        \item Task system for many micro tasks
        \item targets hybrid distributed large scale systems (e.g. Frontier)
        \item Available on all HPC systems via the Exascale Computing Project packages
        \end{itemize}
      \end{minipage}
    \end{column}\begin{column}{.7\linewidth}
      Support for task-based expression and execution in HPC
      \begin{itemize}
      \item How to manage asynchronous progress (of work, of communications)
      \item Promote separation of concerns: Algorithm expresses maximum parallelism, runtime maps to hardware
      \item Manage data hosting transparently (move data between nodes or devices in the background)
      \end{itemize}

      \bigskip

      \scriptsize PaRSEC software
      \begin{itemize}
      \item Provides various interfaces to program task systems (DSL, traditional 'insert-task' API, templated C++ interface for dynamic / computation-dependent DAGs)
      \item Comes with an ecosystem of tools (performance visualization, debugging)
      \end{itemize}
    \end{column}
  \end{columns}
  
\end{frame}

\begin{frame}
  \frametitle{My contributions related to PaRSEC}
  
  My major contributions related to \textcolor{brown}{PaRSEC}:
  \begin{itemize}
  \item Co-designer of the core engine
  \item Dynamic task schedulers, accelerator management
  \item Programming Interfaces 
    \begin{itemize}
    \item (Parameterized Task Graphs; Dynamic Task Discovery; Template Task Graphs)
    \end{itemize}
  \item Distributed Algorithms using or for PaRSEC
    \begin{itemize}
    \item Termination detection 
    \item Data redistribution
    \item Dense Linear Algebra
    \item Sparse Tensor Contraction -- \textcolor{red}{Application to Quantum Chemistry} 
    \end{itemize}
  \item Soft errors resilience in Task Systems
  \item Profiling and performance analysis ecosystem
  \end{itemize}
  
  PaRSEC software available by default on all HPC systems via the ECP packaging
  
  Significant impact on multiple communities (HiCMA -- \textcolor{red}{Gordon Bell finalist in 2022 with PaRSEC}; NWChemEX; MADNESS; MPQC; TiledArray; Chameleon; PASTiX)
    
\end{frame}

\section*{Resp.}

\begin{frame}
  \frametitle{Administrative and Pedagogical Responsibilities; Research Management}

  \begin{columns}
    \begin{column}{.45\linewidth}

      \scriptsize
      Research Management:
      \begin{myitemize}
        \myitem{PI of 2 NSF projects (\$2.5M), 1 ANR project}
        \myitem{CoPI of 1 STREP project}
        \myitem{CoPI of 3 NSF projects (\$2.7M)}
        \myitem{CoPI of 1 DoE project (\$4.7M)}
      \end{myitemize}

      \bigskip
      
      Advising (Master Interns, PhD students, software engineers and postdoc researchers):
      \begin{myitemize}
      \myitem{In France: 3 PhD students, 1 software engineer, 3 Master students}
      \myitem{In the USA: No official role of advisor in the USA, due to
        lack of tenure (policy of UTK). I worked closely with 7 PhD students, 3
        postdoctoral researchers, 8 master students (including some from
        ENS Lyon)}
      \end{myitemize}
  
%  thesards US: Qinglei Cao, Yu Pei, Reazul Hoque, Peng Du, Chongxiao Cao, Wesley Bland, Teng Ma
%  thesards FR: Camille Coti, Olivier Peres, Fatiha Bouabache
%  Ingenieur logiciel: Eric Rodriguez, 
%  postdoc: Joseph Schuchart, Amina Guermouche, Ala Rezmerita, Francois Lesueur, Damien Genet
%  Stagiaires M2 (ENS): Valentin Le Fevre, Julien Herrmann
%  Stagiaires M2 ou autre non ENS: Chunyan Tang, Hinde-Lilia Bouziane, Boris Collin, Michael Cadilhac, Vincent Bridonneau, Peter Gaultney

      \bigskip

      Pedagogical Responsibilities (as MdC in Paris-XI): year 1 to 3 of
      'ing\'enieurs par alternance' cursus for the informatics section -- class 2005, 2006, 2007
    \end{column}\begin{column}{.45\linewidth}\scriptsize
      Pedagogical Activities while at UTK:
      \begin{myitemize}
      \myitem{1 Research School -- 1 week $8\times 3h$}
      \myitem{13 Tutorials on Fault-Tolerance in HPC -- 1 day 8h}
      \myitem{4 Tutorials on task programming -- 1/2 day 4h}
      \end{myitemize}

      \bigskip

      Participation in committees:
      \begin{myitemize}
      \myitem{Executive Director for UTK in the JLESC since 2023}
      \myitem{Member of the directors committee of ICL at UTK, since 2014}
      \myitem{Member of the hiring committee (CSSU or CS) from Universit\'e
        Paris-XI from Sept 2006 to May 2010}
      \myitem{NSF Panelist in 2018, 2019}
      \end{myitemize}

      \bigskip

      Editorial responsibilities: Associate Editor of Journal of
      Parallel and Distributed Computing since 2022

    \end{column}
  \end{columns}
 
\end{frame}

\section{Research Program and Integration in ROMA}

\begin{frame}
  \frametitle{Task-based approaches $\Rightarrow$ sea-change in fault tolerance}

  \begin{center}
    \begin{tabular}{C{.4\linewidth}|C{.4\linewidth}}
      General-purpose FT & Application-specific FT \\\hline
      Coordinated checkpointing and rollback-recovery & Algorithm-Based Fault Tolerance  \\
      Replication                                     & Iterative convergence
    \end{tabular}

    \medskip
    
    \textcolor{blue}{HPC FT approaches have been designed for bulk-synchronous programming}
    
    \bigskip
  
    Task-based approach: asynchronous execution\\
    and dynamically scheduled independent tasks

    \bigskip
  
    \textcolor{red}{We need to redesign fault-tolerance in the context of task systems}
  \end{center}

\end{frame}

\frame{
  \frametitle{Opportunity and Challenge: Malleability}

  \begin{beamerboxesrounded}{Opportunity}
    \begin{center}
      \textcolor{red}{Task systems introduce generic malleability}
    \end{center}

    \begin{itemize}
    \item Tasks are units of work with identified inputs and outputs
    \item Runtime systems can migrate tasks to react to external condition changes (esp. process crash)
    \end{itemize}
  \end{beamerboxesrounded}

  Challenges:
  \begin{itemize}
  \item Non-linear performance degradation due to attrition
    \begin{itemize}
    \item[\frownie] Work and data placement is critical to algorithm performance and usually left to the programmer
    \item[\smiley] Crashed processes should be replaced, so performance degradation should only be transient
    \end{itemize}
  \item Redesign checkpointing as a per-task basis
    \begin{itemize}
    \item noncoordinated checkpointing and message logging
    \item selective replication
    \end{itemize}
  \end{itemize}
}

\frame{
  \frametitle{Applying ABFT to Task-Based Runtime Systems}

  \begin{itemize}
  \item ABFT: bulk-synchronous 'by nature'
    \begin{itemize}
    \item Need to keep the cheksum valid at each step
    \item Efficient \emph{because} checksum is updated by only extending synchronous operation
    \end{itemize}
  \item Task-Based Runtime Systems: asynchronous 'by nature'
    \begin{itemize}
    \item Add strict control flow to keep checksum up to date? \frownie
    \item How to react to failures? Conditional DAG?
    \end{itemize}
  \item ABFT + Tasks:
    \begin{itemize}
    \item Extend DAG with checksum / checksum update tasks
    \item Introduce scheduling priorities to make sure checksums are updated
    \item Define alternative paths to produce data, associate cost with paths
      \begin{itemize}
      \item A block can be computed by inversing the checksum
      \item but preferable path is to update directly
      \end{itemize}
    \end{itemize}
  \end{itemize}

  \begin{center}
    \textcolor{red}{Likely collaborations in NUMPEX}
  \end{center}
}

\begin{frame}
  \frametitle{Silent Data Corruption}

  SDCs are not well evaluated\footnote{In general we need more studies on faults in leadership systems and high-end systems}

  Intuition:
  \begin{itemize}
  \item They happen more often as the amount of memory grows
  \item They happen more often as the computational intensity grows
  \end{itemize}

  If the intuition is right, they will happen more and more frequently

  Application-specific fault tolerance can help
  \begin{itemize}
  \item Not necessarily (or not only) to correct failures
  \item But also to detect failures
  \end{itemize}

  \bigskip

  Challenge: Validation of computation is possible but \textcolor{red}{not reliable}

  \textcolor{red}{Ongoing work: how to (efficiently) use unreliable validation of work}

\end{frame}

\begin{frame}
  \frametitle{Fault tolerance as a tool for energy saving}

  \begin{itemize}
  \item Energy consumption of HPC centers becomes a problem (both economical and ecological)
    \begin{itemize}
    \item Current policy is to not bill allocation if a node within the allocation fails
    \item Some platforms are switching to bill-per-Joule
    \item Does it make sense to get 'free' Joules thanks to a node failure?
    \item Cloud-HPC convergence: CPUs are billed as soon as allocated, even if the application cannot be deployed
    \end{itemize}
  \item Using stranded energy in Cloud or HPC
    \begin{itemize}
    \item Non-dispatchable renewable energy sources to power HPC
    \item $\Rightarrow$ Power cost varies with time (soft version)
    \item $\Rightarrow$ Power capping varies with time (hard version)
    \item $\Rightarrow$ Job scheduler may have to kill apps. Resilience can help.
    \end{itemize}
  \end{itemize}
  \begin{center}
    \textcolor{red}{Fits well the Energy Minimization topic in ROMA}
  \end{center}
\end{frame}



\documentclass[compress,aspectratio=169]{beamer}
\usepackage[utf8]{inputenc}
\usepackage{xcolor,soul}
%
\usepackage{xspace}
\usepackage[most]{tcolorbox}
\newtcolorbox{myblock}[2][]{beamer,title=#2,fonttitle=\sffamily,
  left=1mm,right=1mm,top=1mm,bottom=1mm,arc=2mm,
  colback=black,colupper=white,colframe=yellow,
  coltitle=black,title style={top color=red!70,bottom color=yellow},
  #1}

\mode<presentation>
{
  \usetheme{Replica}
  \setbeamercovered{transparent}
}

\AtBeginSection[]
{
  \begin{frame}<beamer>
    \frametitle{Outline}
     {\tiny
       \tableofcontents[sectionstyle=show/shaded,subsectionstyle=show/hide/hide]
    }
  \end{frame}
}

\title[Rencontre avec le DI]{Projet d'Enseignement}
\subtitle{S\'eminaire de Recrutement\\
  Laboratoire de l'Informatique du Parall\'elisme\\
  \'Ecole Normale Sup\'erieure de Lyon}

\author[herault@icl.utk.edu]{Thomas Herault, University of Tennessee, Knoxville}

\date[13 Mai 2024]{}

\begin{document}
\begin{frame}
  \titlepage
\end{frame}

\section{Cours de Syst\`eme d'Exploitation}

\begin{frame}
  \frametitle{Syst\`eme d'Exploitation}

  \begin{beamerboxesrounded}{But du cours}
    \begin{itemize}
    \item Vue d'ensemble du fonctionnement des syst\`emes d'exploitation
    \item Composants physiques d'un ordinateur, interactions, interface
    \item Repr\'esentation abstraite de la machine
    \item Compr\'ehension des co\^uts (m\'emoire ou temps) li\'e aux diff\'erentes op\'erations 
    \end{itemize}
  \end{beamerboxesrounded}

  \bigskip
  
  \small R\'ef\'erences:\\
  $[1]$ Silberschatz, Abraham, and Peter Baer Galvin. "Operatings System Concepts." (2013).\\
  $[2]$ Tanenbaum, Andrew. Modern operating systems. Pearson Education, Inc., 2009.\\
  $[3]$ Arpaci-Dusseau, Remzi H., and Andrea C. Arpaci-Dusseau. Operating systems: Three easy pieces. Arpaci-Dusseau Books, LLC, 2018.
  
\end{frame}

\begin{frame}
  \frametitle{D\'ecoupage du cours (Syst\`emes d'Exploitation)}

  \begin{columns}
    \begin{column}{.45\linewidth}
      \begin{description}
      \item[Introduction] {\footnotesize Histoire des OS, r\^oles, conceptions}
      \item[Processus] {\footnotesize Concept de processus, partage de temps, ordonnancement, communication interprocess}
      \item[Synchronisation] {\footnotesize Gestion des interd\'ependances, exclusion mutuelle, s\'emaphores, priorit\'es}
      \item[M\'emoire] {\footnotesize M\'emoire virtuelle, pagination, politiques d'\'eviction, support mat\'eriel}
      \end{description}
    \end{column}\begin{column}{.45\linewidth}
      \begin{description}
      \item[Stockage] {\footnotesize Gestion des entr\'ees-sorties, interfaces, structures et fonctions internes}
      \item[Syst\`eme de fichiers] {\footnotesize Hi\'erarchie, r\'epertoire, volumes; journalisation, aggr\'egation et fiabilisation des ressources}
      \item[S\'ecurit\'e] {\footnotesize Droits d'acc\`es (fichiers, CPU, m\'emoire), isolation, contr\^ole d'acc\`es, redondance}
      \end{description}
    \end{column}
  \end{columns}

  \smallskip
  
  Sujets additionels: R\'eseaux et syst\`emes distribu\'es (TCP/IP); virtualisation
\end{frame}
  
\section{Projet Int\'egr\'e}

\begin{frame}
  \frametitle{Projet Int\'egr\'e}

\end{frame}

\section{Cours de HPC pour le M2}

\begin{frame}
  \frametitle{Programmation Parall\'ele: enjeux, outils, et recherche}

  \begin{beamerboxesrounded}{But du cours}
    Vue d'ensemble du calcul \`a haute performance (HPC) et des th\'ematiques de recherche d\'evelop\'ees \`a l'ENS dans le HPC
  \end{beamerboxesrounded}

  \bigskip
  
  \small R\'ef\'erences:\\
  $[1]$ Hager, Georg, and Gerhard Wellein. Introduction to high performance computing for scientists and engineers. CRC Press, 2010.\\
  $[2]$ Sterling, Thomas, Maciej Brodowicz, and Matthew Anderson. High performance computing: modern systems and practices. Morgan Kaufmann, 2017.\\
  $[3]$ Dongarra, Jack, Thomas Herault, and Yves Robert. Fault tolerance techniques for high-performance computing. Springer International Publishing, 2015.
  
\end{frame}

\begin{frame}
  \frametitle{D\'ecoupage du cours}
  \begin{columns}
    \begin{column}{.45\linewidth}
      \begin{itemize}
      \item Historique : quel type d'application, \'evolution des architectures
      \item Mod\`eles de programmation: MPI+OpenMP+CUDA
      \item Mod\`eles de programmation: Syst\`emes de t\^aches
      \item Simulation et calcul num\'erique -- alg\`bre lin\'eaire dense et creuse
      \end{itemize}
    \end{column}\begin{column}{.45\linewidth}
      \begin{itemize}
      \item Ordonnancement (jobs, t\^aches, communications)
      \item Tol\'erance aux fautes (approches g\'enerales)
      \item Tol\'erance aux fautes (approches sp\'ecifiques aux applications)
      \item Extension de MPI pour la tol\'erance aux fautes
      \end{itemize}
    \end{column}
  \end{columns}
\end{frame}

\end{document}

\section{Conclusion and Future Work}

\frame{
  \frametitle{Current Practice for Fault Tolerance in HPC}

  Most applications do not have \textcolor{red}{any} fault-tolerance mechanisms

  \textcolor{blue}{Long running applications} mostly implement
  \begin{itemize}
  \item application-level checkpointing
  \item On disk (PFS)
    \begin{itemize}
    \item Often using libraries (VeloC, HDF5, ADIOS2, ...)
    \item Often with local cache / local staging
    \end{itemize}
  \item Coordinated checkpoint and rollback recovery
  \item With full application-level synchronization
  \item More or less periodic (application-dependent)
  \item But not often using the optimal checkpointing period
    \begin{itemize}
    \item MTBF is often unknown, and \emph{allocations} define the period
    \end{itemize}
  \end{itemize}

  \begin{overlayarea}{\linewidth}{0cm}
    \only<2>{%
      \vspace{-6cm}
      \begin{center}
        \begin{minipage}{.7\linewidth}
          \begin{alertblock}{Faults are not a problem}

            Last 10 years:
            \begin{itemize}
            \item MTBF of \textcolor{blue}{systems} appear to remained capped
            \item \# nodes in Top10 has \emph{decreased}
            \item Computing power of nodes has significantly increased thanks to \emph{many GPUs} / node
            \end{itemize}
          \end{alertblock}
        \end{minipage}
      \end{center}
    }
  \end{overlayarea}%
}

\begin{frame}
  \frametitle{Reasons to continue research on FT for HPC}

  \begin{itemize}
  \item Scale remains the adversary
    \begin{itemize}
    \item End of Dennard scaling (ca 2006)
    \item Moore's Law survives (more or less) by doubling cores (manycore)
    \item or thanks to addition of accelerators
    \end{itemize}
  \item \quesley Will we continue to double the MTBF of nodes at the same speed we double their power?
  \item \quesley Market argument: no-one will commission a supercomputer with $MTBF < \sim 12h$
    \begin{itemize}
    \item \frownie See anecdotal evidence with Sunway TaihuLight
    \item \smiley What is the cost of keeping up with the component reliability increase?\\
      What is the cost of introducing software fault tolerance?\\
      (SW\_Qsim, GB 2021, uses a resilient all-reduce because of failures in new Sunway)
    \item \smiley 'the likelyhood for a supercomputer running for weeks on end is approximately zero' -- Jensen Huang's GTC Keynote, March 18, 2024.
    \end{itemize}
  \end{itemize}
\end{frame}

\begin{frame}
  \frametitle{Silent Data Corruption}

  SDCs are not well evaluated\footnote{In general we need more studies on faults in leadership systems and high-end systems}

  Intuition:
  \begin{itemize}
  \item They happen more often as the amount of memory grows
  \item They happen more often as the computational intensity grows
  \end{itemize}

  If the intuition is right, they will happen more and more frequently

  Application-specific fault tolerance might help
  \begin{itemize}
  \item Not necessarily (or not only) to correct failures
  \item Mostly to detect failures
  \end{itemize}

  Overhead is negligible!

  \begin{center}
    \textcolor{red}{Advocating for hardened/trustworthy version of popular computational libraries}
  \end{center}  
\end{frame}

\begin{frame}
  \frametitle{Other reasons to continue research on FT for HPC}

  \begin{itemize}
  \item Energy consumption becomes a problem (both economical and ecological)
    \begin{itemize}
    \item Current policy is to not bill allocation if a node within the allocation fails
    \item Some platforms are switching to bill-per-Joule
    \item Does it make sense to get 'free' Joules thanks to a node failure?
    \item[] ~
    \item Cloud-HPC convergence: CPUs are billed as soon as allocated, even if the application cannot be deployed
    \end{itemize}
  \item Using stranded energy in Cloud/HPC
    \begin{itemize}
    \item Non-dispatchable renewable energy sources to power HPC
    \item $\Rightarrow$ Power cost varies with time (soft version)
    \item $\Rightarrow$ Power capping varies with time (hard version)
    \item $\Rightarrow$ Job scheduler may have to kill apps. Resilience can help.
    \end{itemize}
  \end{itemize}
\end{frame}

\frame{
  \frametitle{Resilience Techniques Applied to Manage Stragglers}

  \begin{itemize}
  \item Stragglers (slow processes) $\sim$ Failed Processes
    \begin{itemize}
    \item (In fully asynchronous systems, they are indistinguishable)
    \item ABFT techniques: obtain $n-k$ results over $n$ to decide
    \item Extend MPI/ULFM for relaxed collectives / cancellable sends?
      \begin{itemize}
      \item e.g. (All)Reduce with only $k < np$ participants?
      \item e.g. Discard contributions from slow processes?
      \end{itemize}
    \end{itemize}
  \end{itemize}
}

\frame{
  \frametitle{Applying Algorithm Based Fault Tolerance to TBRS}

  \begin{itemize}
  \item ABFT: bulk-synchronous 'by nature'
    \begin{itemize}
    \item Need to keep the cheksum valid at each step
    \item Efficient \emph{because} checksum is updated by only extending synchronous operation
    \end{itemize}
  \item TBRS: asynchronous 'by nature'
    \begin{itemize}
    \item Add strict control flow to keep checksum up to date? \frownie
    \item How to react to failures? Conditional DAG?
    \end{itemize}
  \item ABFT + TBRS:
    \begin{itemize}
    \item Extend DAG with checksum / checksum update tasks
    \item Introduce scheduling priorities to make sure checksums are updated
    \item Define alternative paths to produce data, associate cost with paths
      \begin{itemize}
      \item A block can be computed by inversing the checksum
      \item but preferable path is to update directly
      \end{itemize}
    \end{itemize}
  \end{itemize}
}

\frame{
  \frametitle{Optimal Checkpointing and Task-Based Runtime Systems}
  
    Scheduling checkpoints optimally in TBRS is NP-complete!

    Yet, a TBRS is a parallel application

    And we know what is the \emph{optimal} checkpoint period of a parallel application.
}

\frame{
  \frametitle{Looking back at hierarchical/noncoordinated checkpointing}
  
    'Narrow Gap' of applicability for hierarchical checkpointing

    Advantages of noncoordinated checkpointing:
    \begin{itemize}
    \item no synchronization overheads (\quesley -- but communications are impacted)
    \item checkpoints are staggered (\quesley -- but staging-out of checkpoints works in coordinated)
    \item only failed processes need to restart (\quesley -- but surviving processes need to wait in tightly-coupled applications)
      \begin{itemize}
      \item \smiley noncoordinated should be revisited for \emph{moldable} applications
      \item \smiley TBRS with work migration are ideal candidates
      \end{itemize}
    \end{itemize}
}

\frame{
  \frametitle{Future Work}

  \begin{itemize}
  \item Energy! Silent Errors!
  \item Stragglers
  \item ABFT + TBRS
  \item Optimality of checkpointing in TBRS
  \item Noncoordinated Checkpointing in TBRS
  \end{itemize}

  \vspace{\fill}
  
  \centering Thank you

  \vspace{\fill}
  
  \centering Questions?

  \vspace{\fill}

    \begin{overlayarea}{\linewidth}{0cm}
    \only<2>{%
      \vspace{-7cm}
      \begin{center}
        \begin{minipage}{.9\linewidth}
          \begin{block}{Nothing about AI !?!}

            Roll-forward approaches are probably interesting for AI applications:
            \begin{itemize}
            \item Relaxed collectives / stragglers management
            \item Completely ignore slow processes contribution?
            \end{itemize}

            Using Deep Learning / Reinforced Learning:
            \begin{itemize}
            \item Place checkpoints in TBRS?
            \item Generate missing data for iterative/converging applications
            \item Silent Data Corruption Detector?
            \end{itemize}
          \end{block}
        \end{minipage}
      \end{center}
    }
  \end{overlayarea}%

}

\miniframesoff
\section*{}




\end{document}
