%%
%% TODO: faire apparaitre les problematiques de recherche
%%

\frame{
    \frametitle{Background}

    \pgfplotsset{
        /pgf/number format/year/.style={fixed, 1000 sep={}}
    }

    \resizebox{\textwidth}{!}{
    \begin{tikzpicture}
        \draw[->,line width=.2mm,ultra thick] (0,0) -- (25,0);
        \foreach \x in {0,...,24}{\draw[thick] (\x,0.2)--(\x,-0.2);}
        \foreach \x in {0,4,...,24}{\pgfmathsetmacro\result{\x + 2000}\node at (\x,-0.4) {\pgfmathprintnumber[year]{\result}};}

        \fill[rounded corners=10pt,fill=black,fill opacity=0.2] (0,0.3) rectangle (3,-0.3);
        \node[align=center,text width=3cm] at (1.5,1) {Moniteur Normalien (Paris XI)};

        \fill[rounded corners=10pt,fill=black,fill opacity=0.2] (3,0.3) rectangle (4,-0.3);
        \node[align=center,text width=2cm] at (3.5,1) {ATER (Paris XI)};

        \fill[rounded corners=10pt,fill=black,fill opacity=0.2] (4,0.3) rectangle (24,-0.3);
        \node[align=center,text width=8cm] at (14,1) {Ma\^\i{}tre de Conf\'erences (Paris XI)};

        \fill[rounded corners=10pt,fill=black,fill opacity=0.2] (8,-0.7) rectangle (10,-1.3);
        \node[align=center,text width=3cm] at (9,-1) {D\'et. INRIA};

        \fill[rounded corners=10pt,fill=black,fill opacity=0.2] (10,-0.7) rectangle (20,-1.3);
        \node[align=center,text width=6cm] at (15,-1) {D\'etachement UTK};

        \fill[rounded corners=10pt,fill=black,fill opacity=0.2] (20,-0.7) rectangle (24,-1.3);
        \node[align=center,text width=3cm] at (22,-1) {Disponibilit\'e};

        \fill[rounded corners=10pt,fill=black,fill opacity=0.2] (8,-1.5) rectangle (10,-2.1);
        \node[align=center,text width=3cm] at (9,-1.8) {Vis. Sch. (UTK)};

        \fill[rounded corners=10pt,fill=black,fill opacity=0.2] (10,-1.5) rectangle (17,-2.1);
        \node[align=center,text width=6cm] at (13.5,-1.8) {Research Scientist II (UTK)};

        \fill[rounded corners=10pt,fill=black,fill opacity=0.2] (17,-1.5) rectangle (20,-2.1);
        \node[align=center,text width=3cm] at (18.5,-1.8) {Res. Dir. (UTK)};
    
        \fill[rounded corners=10pt,fill=black,fill opacity=0.2] (20,-1.5) rectangle (24,-2.1);
        \node[align=center,text width=4cm] at (22,-1.8) {Res. Ass. Pr. (UTK)};

        \fill[rounded corners=10pt,fill=red,fill opacity=0.2] (0,-2.5) rectangle (6,-3.1);
        \node[align=center,text width=6cm] at (3,-2.8) {Self-Stabilization, Dist. Alg.};

        \fill[rounded corners=10pt,fill=black,fill opacity=0.2] (7,-2.5) rectangle (11,-3.1);
        \node[align=center,text width=6cm] at (9,-2.8) {``D\'efi S\'ecurit\'e''};

        \fill[rounded corners=10pt,fill=purple,fill opacity=0.2] (13,-2.5) rectangle (24,-3.1);
        \node[align=center,text width=12cm] at (18.5,-2.8) {Performance Modeling for Fault Tolerance};
       
        \fill[rounded corners=10pt,fill=green,fill opacity=0.2] (3,-3.3) rectangle (10,-3.9);
        \node[align=center,text width=6cm] at (6.5,-3.6) {Rollback recovery in MPICH};

        \fill[rounded corners=10pt,fill=brown,fill opacity=0.2] (10,-3.3) rectangle (24,-3.9);
        \node[align=center,text width=18cm] at (17,-3.6) {Task-Based Runtime Systems -- Prog. Inter.; Applications; Mem. Const. Alg.};

        \fill[rounded corners=10pt,fill=blue,fill opacity=0.2] (8,-4.1) rectangle (24,-4.7);
        \node[align=center,text width=12cm] at (16,-4.4) {Fault Tolerant MPI, Application-Specific FT for HPC};

        \fill[rounded corners=10pt,fill=black,fill opacity=0.2] (2,-4.1) rectangle (7,-4.7);
        \node[align=center,text width=6cm] at (4.5,-4.4) {App. Prob. Model Checking};
    \end{tikzpicture}}
}   

\frame{
  \frametitle{My contributions to HPC (overview)}

  Two complementary approaches to Fault Tolerance
  \begin{itemize}
  \item General Purpose Fault Tolerance
    \begin{itemize}
    \item Checkpointing and rollback-recovery
      \begin{itemize}
      \item Experimental approach: implementing new protocols and evaluating them in 'real' applications / deployments
      \item Performance models approach: design mathematical models to project performance at scale or under new assumptions
      \end{itemize}
    \end{itemize}
  \item Application-Specific Fault Tolerance
    \begin{itemize}
    \item Design / adapt communication middlewares to write fault-tolerant applications
    \item Design new fault-tolerance schemes for parallel applications
    \end{itemize}
  \end{itemize}

  Task-Based Runtime Systems and Fault Tolerance
  \begin{itemize}
  \item New runtime systems and programming interfaces for modern hardware architectures
  \item Design fault-tolerant applications over TBRS
  \end{itemize}
}

\frame{
  \frametitle{Overview: General Purpose Fault Tolerance}

  \begin{columns}
    \begin{column}{.5\textwidth}
      Experimental approach: MPICH-V
      \begin{itemize}
      \item Coordinated Rollback-Recovery
        \begin{itemize}
        \item Blocking/Non Blocking
        \end{itemize}
      \item Uncoordinated Rollback-Recovery
        \begin{itemize}
        \item Message Logging
        \item Optimistic / Pessimistic / Causal
        \end{itemize}
      \item Hierarchical Approach
      \end{itemize}
    \end{column}
    \begin{column}{.5\textwidth}
      Performance Models for Rollback-Recovery
      \begin{itemize}
      \item \textcolor{blue}{Young-Daly}
      \item \textcolor{blue}{Hierarchical Checkpointing}
      \item In-Memory Checkpointing
      \item Managing I/O Contention
      \end{itemize}
    \end{column}
  \end{columns}
}

\frame{
  \frametitle{Overview: Application-Specific Fault-Tolerance}
  
  \begin{columns}
    \begin{column}{.5\textwidth}
      User-Level Failure Mitigation
      \begin{itemize}
      \item Specification / Standardization
      \item \textcolor{blue}{Failure Detection}
      \item \textcolor{blue}{Consensus}
      \item Silent Errors
      \end{itemize}
    \end{column}
    \begin{column}{.5\textwidth}
      Roll-forward fault-tolerance
      \begin{itemize}
      \item ABFT for dense matrix factorization
      \item Composite Approach: ABFT \& Checkpointing
      \item Moldable Applications
      \end{itemize}
    \end{column}
  \end{columns}
}

\frame{
  \frametitle{Overview: Task-Based Runtime Systems}
  
  \begin{columns}
    \begin{column}{.5\textwidth}
      Parallel Runtime Scheduler and Execution Controller (PaRSEC)
      \begin{itemize}
      \item Micro task (task scheduling overhead $\sim 200ns$)
      \item Distributed (runs with thousands of nodes)
      \item Hybrid (Intel, AMD, NVIDIA GPUs)
      \item Implicit data movement (from device to device, in the background)
      \item Multiple programming interfaces
        \begin{itemize}
        \item Parameterized Task Graphs
        \item Dynamic Task Discovery
        \item Template Task Graphs
        \end{itemize}
      \end{itemize}
    \end{column}
    \begin{column}{.5\textwidth}
      \begin{itemize}
      \item Applications:
        \begin{itemize}
        \item Dense Linear Algebra (DPLASMA)
        \item Sparse Linear Algebra (PaStiX, TiledArray)
        \item Quantum Chemistry (NWChem, MPQC)
        \item Multi-representation (HiCMA)
        \end{itemize}
      \item Node-to-node migratable tasks
      \item Resilient extensions
        \begin{itemize}
        \item Resilient to Silent Data Corruptions via Algorithm-Based Fault-Tolerance validation
        \item Task-based checkpoint and partial rollback-recovery
        \end{itemize}
      \end{itemize}
    \end{column}
  \end{columns}
}
